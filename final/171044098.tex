\documentclass{beamer}
\usetheme{Boadilla}
\usepackage{amsmath}

\title{MATH 118 Probability and Statistics }
\subtitle{Final Presentation}
\author{171044098 - Akif Kartal}
\institute{Gebze Technical University}

%\usetheme{lucid}
\begin{document}
	\frame {
		\titlepage
	}
	\frame{
		\frametitle{Topics to be covered and Why?}
		\section{Topics}
		\subsection{Probability of an Event}
		\subsection{Conditional Probability}
		\section{Why did I choose these topics?}
		\subsection{When I analyze our topics along the semester, I saw that these two topics are very important topics for probability, they are like \textbf{backbone of the probability} and lastly they are frequently used topics in real life. Also, I have chose 2 different topic, because first we need to understand what is probability, then we can study conditional probability.}
		\tableofcontents
		\begin{itemize}
		\item Let Start with Probability of an Event.
		\end{itemize}
	}
	\frame{
	    \frametitle{Probability of an Event}
	    \begin{block}{Event}
		Event is an outcome or occurrence that has a probability assigned to it.
		\end{block}
		\begin{block}{Probability of an Event}
		\[
 		  Probability = \frac{\text{The number of wanted outcomes}}{\text{The number of possible outcomes}}
		\]
		\newline
		\newline
		If an experiment can result in any one of N different equally likely outcomes, and if exactly n of these outcomes 			         correspond to event A, then the probability of event A is
		\newline
		\[P(A) = \frac{\text{n}}{\text{N}}, \] \[0 \leq P(A) \leq 1.\]
		\end{block}
	}
	\frame{
	    \begin{example}
			Each of the letters HELLO is written on a card. A card is chosen at random from the bag. What is the probability of 				getting the letter 'L'?
		\end{example}
		\begin{alertblock}{Solution}
		Since the card is randomly selected, it means that each card has the same chance of being selected. The sample space for this experiment is;
		\newline
		\[S = \lbrace H, E, L_{1} , L_{2} , O \rbrace\]
		\newline
		There are two cards with the letter 'L'.
		\newline
		\[\text{Let A = event of getting the letter 'L'} = \lbrace  L_{1} , L_{2} \rbrace\]
		\[P(A) = \frac{2}{5} \]
		\end{alertblock}
	}
	\frame{
	    \begin{example}
			The names of four directors of a company will be placed in a hat and a 2-member delegation will be selected at random to represent the company at an international meeting. Let A, B, C and D denote the directors of the company. What is the probability that \\
			\begin{enumerate}[i]
			\item A is selected?
			\item A is not selected?
			\end{enumerate}
		\end{example}
		\begin{alertblock}{Solution(i)}
		Firstly, The sample space for this experiment is;
		\[S = \lbrace AB, AC, AD, BC, BD, CD \rbrace\] 
		When we choose A, we must choose one of the remaining 3 directors to go with A. There are;
		\[{4 \choose 2} = 6\]
		possible combinations. 			
		\end{alertblock}
	}
	\frame{
	\begin{alertblock}{Solution(i) cont.}
		Then, the probability that A is selected is;
		\[\frac{{1 \choose 1} \times {3 \choose 1}}{{4 \choose 2}} = \frac{3}{6} = \frac{1}{2}\]		
		\end{alertblock}
		\begin{alertblock}{Solution(ii)}
		We found A is selected is $\frac{1}{2}$, then A is not selected is;
		\[1 - \frac{1}{2} = \frac{1}{2}\]		
		\end{alertblock}
	}
	\frame{
	\frametitle{Conditional Probability}
	\begin{block}{Conditional Probability}
		Conditional Probability is a probability which measure the probability of one event occurring relative to another occurring.
	\end{block}
	\begin{block}{How to express?}
		If we want to express the probability of one event happening given another one has already happened, we use the $"\mid"$ symbol to mean "given", and we say; 
		\newline
		\newline
		$P(A \mid B) = \text{The probability of A given that we know B has happened.}$
		\end{block}
	}
	\frame{
		\begin{definition}
		If A and B are two events in a sample space S, then the conditional probability of A given B is defined as
		\[P(A \mid B) = \frac{P(A \cap B)}{P(B)} \hspace{0.3cm} \text{when} \hspace{0.2cm} P(B) > 0. \]	
		\end{definition}
	}
	\frame{
	\begin{example}
		Netflix says that(approximately);
		\begin{itemize}
		\item 10,234,231 people watched Zootopia movie on Netflix
		\item 3,110,153 people watched both Zootopia and Monsters movies on Netflix
		\end{itemize}
		What is the probability that a user will watch Zootopia, given that he/she watched Monsters?
	\end{example}
	\begin{figure}
	\includegraphics[width=3cm, height=3.5cm]{net.jpg}
	\hspace{0.5cm}
	\includegraphics[width=3cm, height=3.5cm]{zoo.jpg}
	\hspace{0.5cm}
	\includegraphics[width=3cm, height=3.5cm]{mos.jpg}
	\end{figure}
	
	}
	\frame{
		\begin{alertblock}{Solution}
		\[P(A \cap B)= \text{people who watched both Zootopia and Monsters on Netflix}\]
		\[P(B)= \text{people who watched both Zootopia and Monsters on Netflix}\]	
		\newline
		\[P(A \mid B) = \text{?}\]
		\newline
		\[P(A \mid B) = \frac{P(A \cap B)}{P(B)} = \frac{3,110,153}{10,234,231} = 0.30\]
		\end{alertblock}
	}
	\frame{
	\frametitle{Independence}
	\begin{definition}
		Events A and B are independent, if information about one does not affect the other. This is;
		\[P(A \mid B) = P(A) \]
		\[P(B \mid A) = P(B) \]
		This is equivalent to, events A and B are independent if and only if
		\[P(A \cap B) = P(A)P(B) \]
		Therefore, to obtain the probability that two independent events will both occur, we simply find the product of their individual probabilities.	
		\end{definition}
	}
	\frame{
	\frametitle{Multiplication Rule}
	\begin{definition}
		This rule follows directly from the definition of conditional probability;
		\[P(A \cap B) = P(A)P(B \mid A) \]
		or
		\newline
		\[P(A \cap B) = P(B)P(A \mid B) \]
	\end{definition}
	\begin{example}
			What is the probability that two female students will be selected at random to participate in a certain research project, from a class of 7 males and 3 female students?
	\end{example}
	}
	\frame{
		\begin{alertblock}{Solution}
		First we need to define events; Let;
		\newline
		\newline
		A = the first student selected is a female
		\newline
		B = the second student selected is a female
		\newline
		\[P(A \cap B) = P(A)P(B \mid A) = \frac{3}{10} \times \frac{2}{9} = \frac{6}{90}\]
		\[P(A \cap B) = \frac{1}{15}\]
		\end{alertblock}
	}
	\frame{
	\begin{block}{Addition Rule}
		For any two events A and B;
		\[P(A \cup B) = P(A) + P(B) - P(A \cap B) \]
	\end{block}
	\begin{block}{Probability Trees}
		This is a useful device to calculate probabilities when using the probability rules.
	\end{block}
	\begin{columns}
	\column{0.5\textwidth}
	\begin{figure}
		\includegraphics[width=4cm, height=4cm]{test.jpg}
	\end{figure}
	\column{0.5\textwidth}
	\begin{itemize}
		\item We multiply probabilities along the branches.
		\item We add probabilities down columns.
	\end{itemize}
	\end{columns}
	}
	\frame{
	\begin{example}
		Student G wakes up late on average 3 days in every 5 days. 
		\newline
		\newline
		If G wakes up late, the probability G is late for school = $\frac{9}{10}$ 
		\newline
		\newline
		If G does not wakes up late, the probability G is late for school = $\frac{3}{10}$ 
		\newline
		\newline
		On what percent of days does G get to school on time?
		\newline
	\end{example}
	\begin{alertblock}{Solution}
		The probability G wakes up late = $\frac{3}{5}$ 
		\newline
		\newline
		The probability G wakes up on time = $\frac{2}{5}$ 
		\newline
		\newline
		If we draw Probability Tree we will get;
		\newline
	\end{alertblock}
	
	}
	\frame{
	\begin{figure}
		\includegraphics[width=7cm, height=5cm]{test2.jpg}
	\end{figure}
	\begin{alertblock}{Solution cont.}
		Thus, the probability G gets to school on time;
		\[P(G) = \frac{3}{50} + \frac{14}{50} = \frac{17}{50} = \frac{34}{100}\]
		As a result, Student G gets to school on time $34\%$ of the time.
	\end{alertblock}	
	}
	\frame{
	\begin{example}
		A Ph.D. graduate has applied for a job with two universities: A and B. The graduate feels that she has a 60\% chance of receiving an offer from university A and a 50\% chance of receiving an offer from university B. If she receives an offer from
university B, she believes that she has an 80\% chance of receiving an offer from university A.
	\begin{enumerate}[i]
			\item What is the probability that both universities will make her an offer?
			\item What is the probability that at least one university will make her an offer?
			\item If she receives an offer from university B, what is the probability that she will
not receive an offer from university A?
			\end{enumerate}
	\end{example}
	\begin{figure}
		\includegraphics[width=2.5cm, height=3cm]{phd.png}
	\end{figure}
	}
	\frame{
		\begin{alertblock}{Solution}
		From question we have;
		\[P(A) = 0.6\]
		\[P(B) = 0.5\]
		\[P(A \mid B) = 0.8\]
		\begin{enumerate}[i]
			\item $P(A \cap B) = \text{?}$
		\end{enumerate}
		\[P(A \cap B) = P(B)P(A \mid B) \]
		\[P(A \cap B) = 0.5 \times 0.8 = 0.4 \]
		\begin{enumerate}[ii]
			\item $P(A \cup B) = \text{?}$
		\end{enumerate}
		\[P(A \cup B) = P(A) + P(B) - P(A \cap B) \]
		\[P(A \cup B) = 0.6 + 0.5 - 0.4 = 0.7 \]
	\end{alertblock}
	}
	\frame{
	\begin{alertblock}{Solution cont.}
		
		\begin{enumerate}[iii]
\item If she receives an offer from university B, what is the probability that she will not receive an offer from university A?
		\end{enumerate}
		Let;
		\newline
		\newline
		P(D) = receiving offer from university B and not receiving offer from university A.
		\newline
		\newline
		P(B) = receiving offer from university B
		\newline
		\newline
		$ P(A \cap B) = $ receiving offer from university A and receiving offer from university B
		\newline
		\newline
		Then P(D) will be;
		\[ P(D) = P(B) - P(A \cap B) \]
		\[ P(D) = 0.5 - 0.4 = 0.1\]
	\end{alertblock}
	}
	\frame{
	\begin{alertblock}{Solution cont.}
		
		Let;
		\newline
		\newline
		$P(E) = $  not receiving offer from university A, Then;
		\[P(E) = 1 - 0.6 = 0.4\]
		Let;
		\newline
		\newline
		$P(B \mid E)$ = receiving offer from university B $\mid$ not receiving offer from university A
		\newline
		\newline
		Then $P(B \mid E)$ will be;
		\[ P(B \mid E) = \frac{\text{receiving offer from uni. B and not receiving offer from uni. A}}{\text{not receiving offer from university A}} \]
		\[ P(B \mid E) = \frac{P(D)}{P(E)} = \frac{0.1}{0.4} = 0.25 \]
		
	\end{alertblock}
	}
	\frame{
	\frametitle{Summary}
	\begin{itemize}
		\item First, we understand what is probability, event  and probability of an event.
		\item We see some of basic examples of the probability.
		\item Then, we study conditional probability.
		\item With conditional probability, we see Multiplication Rule, Addition Rule and Probability Trees.
		\item Lastly, we have finished with a good example of conditional probability.
	\end{itemize}
	\begin{columns}
	\column{0.65\textwidth}
	\textbf{Thank You...}
	\column{0.35\textwidth}
	\includegraphics[width=4cm, height=3cm]{last.jpg}
	\end{columns}
	}
\end{document}