\documentclass{article}
\usepackage[utf8]{inputenc}
\usepackage[margin=1in,includefoot]{geometry}

% Header and Footer Setup
\usepackage{fancyhdr}
\pagestyle{fancy}
\fancyhead{}
\fancyfoot{}
\fancyfoot[R]{\thepage}
\renewcommand{\headrulewidth}{0pt}
\renewcommand{\footrulewidth}{0pt}
%
%Graphics Setup
\usepackage{graphicx}
\usepackage{float}
\usepackage{subfig}

%list setup
\usepackage{amssymb}
\renewcommand{\labelitemi}{$\blacktriangleright$}
\renewcommand{\labelitemii}{$\bullet$}
\renewcommand{\labelitemiii}{$\circ$}

%Source Code setup
\usepackage{xcolor}
\usepackage{listings}

\definecolor{mGreen}{rgb}{0,0.6,0}
\definecolor{mGray}{rgb}{0.5,0.5,0.5}
\definecolor{mPurple}{rgb}{0.58,0,0.82}
\definecolor{backgroundColour}{rgb}{0.95,0.95,0.92}

\lstdefinestyle{CStyle}{
    backgroundcolor=\color{backgroundColour},   
    commentstyle=\color{mGreen},
    keywordstyle=\color{magenta},
    numberstyle=\tiny\color{mGray},
    stringstyle=\color{mPurple},
    basicstyle=\footnotesize,
    breakatwhitespace=false,         
    breaklines=true,                 
    captionpos=b,                    
    keepspaces=true,                 
    numbers=left,                    
    numbersep=5pt,                  
    showspaces=false,                
    showstringspaces=false,
    showtabs=false,                  
    tabsize=2,
    language=C
}
%


\begin{document}

\begin{titlepage}

	\begin{flushright}
	\textsc{\large April 17, 2021} \\
	\end{flushright}
	\begin{center}
	\Large{\bfseries GTU Department of Computer Engineering \\ MATH118 - Spring 2021 \\ Homework 1 Report  } \\
	\end{center}
	\topskip0pt
	\vspace*{\fill}
	\begin{center}
	\Large{\bfseries Akif Kartal \\ 171044098 }
	\end{center}
	\vspace*{\fill}

\end{titlepage}

\cleardoublepage
\section{Problem Definition}
The problem is to make some operation on covid19 data by using python programming language.
\section{System Requirements}
In order to run program you need to install followings;
\begin{itemize}
	\item Python
	\item openpyxl library for python.
	\item any other library if needed.
	\item \textbf{owid-covid-data.xlsx} file in same directory.
\end{itemize} 
After running wait a few minute program to finish, then \textbf{output.csv} file will be created.
\section{Question 19}
About this question my observations are following; \\ \\
\textbf{Note:} These observations was made with April 16 data.

\begin{enumerate}
  \item First observation is of course number of cases will affect number of death directly.
  \item If the median age is low in a country, then deaths is low.
  \item Economic performance will affect number of vaccinations are administered in each country.
  \item Death rates due to heart disease will \textbf{not} affect total deaths.
  \item Handwashing facilities will \textbf{not} affect total cases.
  	\begin{itemize}
    \item For example Serbia has 97,719 handwashing facilities and Senegal has 20,859 but even tough serbia has two times more population than senegal
    it has 16 times more cases than senegal.
  \end{itemize}
  \item Hospital beds per thousand people will \textbf{not} affect total deaths.
  	\begin{itemize}
    \item For example Japan has 13,05 hospital beds per thousand people and Banglades has 0,8 but total deaths number are almost same. Japan has 9506
    total death and banglades has 10081.
  \end{itemize}
  \item Number of smokers will affect both total cases and deaths.
  	\begin{itemize}
    \item For example Serbia has 37,7 female smokers and 40,2 male smoker and Oman has 0,5 female smokers and 15,6 male smoker even though their
    population is close serbia has 3 times more death and cases than oman.
  \end{itemize}
\end{enumerate}
\end{document}